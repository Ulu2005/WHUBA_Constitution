\documentclass[11pt,a4paper,titlepage]{article} 
\usepackage{ctex} 
\usepackage{fontspec} 
\usepackage{hyperref}
\hypersetup
{
  dvipdfmx,% 设定要使用的 driver 为 dvipdfmx
  unicode={true},% 使用 unicode 来编码 PDF 字符串
  pdfstartview={FitH},% 文档初始视图为匹配宽度
  bookmarksnumbered={true},% 书签附上章节编号
  bookmarksopen={true},% 展开书签
  pdfborder={0 0 0},% 链接无框
  pdftitle={武汉大学棒球协会章程},
  pdfauthor={武汉大学棒球协会},
  %pdfsubject={主题},
  %pdfkeywords={关键词},
  %pdfcreator={应用程序},
  %pdfproducer={PDF 制作程序},% 这个好像没起作用?
  %citecolor=blue,
  %inkcolor=red,
  %anchorcolor=green,
  %urlcolor=blue
 }
%\usepackage{indentfirst} 
%\setlength{\parindent}{2em}
\XeTeXlinebreaklocale "zh"
\XeTeXlinebreakskip = 0pt plus 1pt minus 0.1pt
\renewcommand{\contentsname}{目录}

\title{武汉大学棒球协会章程\\The Constitution Of Wuhan University Baseball Association}
\author{第二版} 
\date{2012年9月1日}

\begin{document} 
\maketitle
%\newpage
\setcounter{tocdepth}{2}
\tableofcontents
\newpage
	\section{总则}武汉大学棒球协会(以下简称协会)是由武汉大学的棒球爱好者发起组织并成立的(始建于1999年,由本校日本留学生和学生创建),以武汉大学学生为主体,接纳武汉地区棒球爱好者参与,是一个旨在在武汉地区推广并发展棒球事业的一个校园、非盈利性组织。我们的终极目标是培养棒球人才,参与全国大学生棒球比赛以及各类全国大赛,并取得优异成绩。
	\section{协会会员的权利和义务}
		\subsection{权利}
			\begin{itemize} 
				\item协会会员享有参加各类比赛、训练和其他集体活动(部分另有说明者除外)的权利
				\item协会会员享有活动期间饮用水、药品等使用权
				\item协会会员享有合理的被训练和被指导权
				\item协会会员享有相关器械的使用权。
			\end{itemize} 
		\subsection{义务}
			\begin{itemize} 
				\item协会会员有遵守本章程的义务
				\item协会会员有服从活动组织者安排的义务
				\item协会会员有按时缴纳会费的义务。
				\item协会会员有向新会员强调安全问题的义务。
			\end{itemize} 
	\section{协会承担的责任}
		\begin{enumerate} 
			\item在已知各种安全问题的前提下,协会不对任何会员的伤病进行负责。
			\item传球、投捕、投打、挥棒等情况下受伤由当事人双方自行进行医疗费协商。
			\item守备练习、比赛等集体活动情况受伤协会承担部分医疗费用\footnote{在非前两者情况下,也就是个人为比赛、训练受伤,无其余责任人。尤指扑垒、盗垒、扑球等情况。会员对自身做出的危险动作可能造成的意外伤害自行负责。}。
		\end{enumerate}  
	\section{协会运作}
		\subsection{职务设定}
		协会设立会长、副会长、经理各一人。
			\subsubsection{会长}
				\begin{enumerate} 
					\item负责协会的日常训练。主要包括训练计划的制定和实施,新老队员技战术的指导等。
					\item负责协会的人员管理。
				\end{enumerate} 
			\subsubsection{经理}
				\begin{enumerate} 
					\item负责协会的财务以及器材统计。
					\item负责每次活动的通知以及人员安排。
					\item负责对外联系、宣传等活动。
					\item负责网络方面运营。包括QQ群,新浪微博,百度贴吧。
				\end{enumerate} 
			\subsubsection{副会长}
				\begin{enumerate} 
					\item主要负责协助会长和经理的工作。
				\end{enumerate} 
		\subsection{附属球队}
		球队队员为协会中坚力量,负责参加各种级别的比赛。新人入会\emph{一年后}可由会长和经理选拔,其余队员同意后入队。
		\begin{itemize} 
			\item球队队员享有器材使用、训练资源使用(包括训练内容安排)的绝对优先权。
			\item球队队员必须严格按照会长的安排进行训练,必须保证出勤率,积极参与训练。
		\end{itemize} 
		\subsection{传统惯例}
		协会每年在毕业季进行新老生毕业对抗赛,赛后聚餐,产生费用由\emph{当年毕业的学生}平均负担。
	\section{安全问题}
	在自身身体状况不佳,以及天色较暗场地环境较差的情况下不得进行各种棒球活动。
		\subsection{传球}
			\begin{enumerate} 
				\item场地中间、传球者身后有人的时候禁止进行传球。
				\item传球时,各组间间距须在2米以上,且必须注意力集中。
				\item传球前一定要确定对方准备妥当,确认后方可传球。
				\item多人传球的时候要按照同一朝向平行排列,不能在人群密集处传球。
			\end{enumerate} 
		\subsection{捕球和投球}
			\begin{enumerate} 
				\item无论何时捕手必须佩戴面罩,否则后果自负。
				\item投打训练时,投手投快速球而非喂球的情况下必须穿戴全部护具。
				\item投手要注意自我保护,投打时小心来球。
			\end{enumerate} 
		\subsection{守备练习}
			\begin{enumerate} 
				\item对于没有把握的球,以保证自身安全为先。
				\item在排队接球的时候,每个队员都要将精神集中在球上,切不可在后面交谈而忽略来球。
				\item接球的队员和其他人保持一定距离,不得闲聊与蹲坐。
				\item漏接的球在飞向其他人活动的区域时要大声喊叫,必须引起其他同学的注意,以保证其他场上活动同学的安全。
				\item新生接地滚球必须使用\emph{软球}。
			\end{enumerate} 
		\subsection{打击练习}
			\begin{enumerate} 
				\item挥棒要在空旷的地方,必须确认身旁5米无人的前提下进行挥棒练习。\emph{队员走动时}也要注意安全。
				\item投打的时候必须佩戴打击头盔,否则不可进行。\emph{软球}则视投手水平再定。
				\item挥棒之后不得甩棒,打击之后将球棒放下再跑垒。
			\end{enumerate} 
		\subsection{跑垒练习}
			\begin{enumerate} 
				\item在武汉大学操场进行练习时,不得扑垒。扑垒练习将另行安排。
			\end{enumerate} 		 
	\section{会费事宜}
		\begin{enumerate}
		\item会费根据入会时间长短,是否在校读书,是否有工作收入进行划分不同金额。每学期由\emph{经理}统一收取。\\
			\begin{table}[htbp]
			\centering
			\begin{tabular} {|c|c|}
				\hline
				成员身份 & 费用/学期 \\
				\hline
				在校生 & 30元 \\
				\hline
				已毕业工作者 & 50元 \\
				\hline
			\end{tabular}
			\end{table}
		\item协会会费主要用于相关器材、药品以及必要的活动经费支出(包括领队的部分通讯费)。
		\item若外出参与比赛,则视情况另行收费。
		\item经费由经理负责,不得借用、挪用及任何形式的脱离相关负责人管辖;负责人应及时更新经费收支情况并详细记录,以备查验;每季度由负责人制“经费收支清单”一份,以群邮件形式发送至各会员的邮箱,以供监督。
		\end{enumerate}
	\section{日常训练}
		\subsection{训练细则}
			\begin{itemize} 
				\item在每次收到活动通知短信后应及时予以回复,以便安排器材及器材搬运人员。
				\item会员必须穿着适宜体育运动的服装,以避免运动伤害。穿运动鞋,以钉鞋为佳。禁止穿牛仔裤、板鞋、篮球鞋。
				\item训练时不得无故离场,离场需向本次训练负责人申请。
				\item热身跑圈:夏季1-2圈;冬季2-3圈。准备活动时间不少于20分钟。
				\item每次训练自备饮用水。
				\item活动结束后会员有义务检查场上情况,检查有没有器材遗漏。
			\end{itemize} 
		\subsection{新人培育}
			\begin{itemize} 
				\item基本传球热身练习的时候以新老生搭配组合,新生要主动向老会员多学习交流,主动认识。
				\item垒间传球分为新老两组,一组为可以自如进行垒间传球的队员,一组为对传球还不熟练的同学,老生多出的人去带领新生练习,指导动作。
				\item守备练习的时候分成两组练习,分组上场,不上场的一组则在场边传球。
			\end{itemize} 
		\subsection{老生义务}
			\begin{itemize} 
				\item老会员必须将培养新会员作为自己的义务牢记于心,教授新会员传球、接球、挥棒以及安全问题。
				\item来参与活动的\emph{协会临时成员}亦应将此作为己任。
			\end{itemize} 
	\section{器材管理}
		\subsection{器材保养}
			\begin{enumerate} 
				\item使用协会公用手套的时候必须佩戴白色线手套,如果没有可以向经理索取。
				\item严禁拿硬球砸墙、砸地。
				\item垒球棒、棒球棒、教练棒不能混用。
				\item手套、头盔要轻拿轻放,结束训练时要整齐摆放。
				\item球要统一摆放于一处,避免丢失。
				\item会员有义务协助整理器材。
			\end{enumerate} 
		\subsection{器材搬运}
			\begin{enumerate} 
				\item原则上会安排距离器材安放地点近的同学拿器材去活动场地,较远的同学也应该轮流搬运器材。
				\item建议每个会员都自己购买的手套,需要意见可以向协会其余成员咨询。
			\end{enumerate} 
		\subsection{器材遗失、损坏}
			\begin{enumerate} 
				\item器材遗失或损坏应该及时向经理汇报,参照原价,适当折旧赔偿。
				\item当器材破损、老化的时候必须及时汇报,以便于维护。
				\item私自外借器材必须向经理登记报备\emph{(需学生证等有效证件,归还时退还学生证)},若遗失或损坏照价赔偿。
			\end{enumerate} 
		\subsection{器材保管}
			\begin{enumerate} 
				\item会员在住宿条件允许的前提下有义务帮助协会存放器材。
				\item器材分配须登记写明,会员保管时有义务保证器材完好。
			\end{enumerate} 
		%\footnote{本文由\LaTeX生成}
\end{document} 
 